\chapter{A client-side solution to session fixation and session hijacking}\label{fixation-solution}

In this chapter, we propose a client-side solution to the \gls{session fixation} and \gls{session hijacking} attacks described in section \ref{fixation}. The proposed solution was also submitted as a paper, co-authored by Philippe De Ryck, Nick Nikiforakis, Lieven Desmet and Frank Piessens, to the W2SP 2011 conference \cite{Bonne2011}. This chapter includes some of their wordings and data, which are indicated by citing the submitted paper. The original version of the paper is attached to this thesis as Appendix \ref{paper}. A scientific poster describing the solution is attached as Appendix \ref{poster}.

The reason for developing a solution at the client side is that the user incentive for using a secure web application is often larger than the web developer incentive for creating one \cite{Johns2011}. To our knowledge, there exists no other practical client-side solution to the session fixation attack \cite{Bonne2011}.

Out of the methods for injecting a \gls{session id} described in section \ref{injecting-sid}, we consider \gls{xss} and \texttt{<meta>}-tag injection the most important. These injection attacks have a high severity rating \cite{Williams2010} and lots of websites are vulnerable \cite{Brown2010}. Similarly, out of the methods for capturing a session ID (see section \ref{capturing}), we consider XSS attacks to be in scope. Ideally, these problems would be solved by finding a complete solution to XSS attacks. Unfortunately, as we saw in section \ref{xss-countermeasures}, solutions to XSS are often lacking, and few have been developed that work exclusively at the client side. Other attack vectors, such as URL rewriting, subdomain cookie setting, \gls{mitm} attacks and response splitting are considered out of scope because either a client-side solution would be unable to distinguish between legitimate SIDs and forged SIDs, or because they exploit a bug at the browser or proxy level.

We first develop a client-side policy that counters session fixation attacks. Afterwards, we implement this policy as a Firefox add-on, and we provide a thorough evaluation. Lastly, we describe how the add-on was extended to provide a solution to the session hijacking attack.

\section{Principle}\label{principle}

The reasoning behind the developed client-side solution is that session IDs will never be set over an untrusted channel, only to be requested over a trusted channel later on. We consider \gls{http} to be trusted, since this channel is controlled entirely by the web server. As an untrusted channel we consider elements in the web page itself, such as JavaScript and \texttt{<meta>} tags, because they often contain user input. Thus, the assumption made is that most websites will set their session identifiers via HTTP, and that websites that do not will never request the SID via this channel. As we will see in section \ref{evaluation}, this assumption is valid for all practical use.

The solution has the form of a proxy that is located at the client-side. As a basic policy, we choose to only allow cookies in outgoing HTTP requests if they were previously set via an HTTP response from the server. This policy is depicted in Figure \ref{fig:clientside-proxy}. When a new cookie is sent to the client via HTTP, the proxy remembers this cookie. When an outgoing request is sent to the server, the proxy checks all outgoing cookies. If one of these was not set via HTTP, it is removed from the request. This prevents all cookies set via JavaScript or \texttt{<meta>} tags from being used over HTTP.

\begin{figure}[htb]
	\centering
	\subfloat[An incoming response containing a new SID.]{
		\includegraphics[width=.7\textwidth]{img/clientside-proxy-1.png}
		\label{fig:clientside-request}
	}\\
	\subfloat[An outgoing request containing a cookie set via HTTP, and one set via JavaScript. The JavaScript cookie is stripped from the request by the proxy.]{
		\label{fig:clientside-response}
		\includegraphics[width=0.7\textwidth]{img/clientside-proxy-2.png}
	}
	\caption{Client-side solution to session fixation}
	\label{fig:clientside-proxy}
\end{figure}

We want to apply this basic policy only to \glspl{session cookie}, instead of all cookies. To see why, consider for example the scenario where a user can set the theme of the current web page by clicking a button\footnote{The website \url{http://www.last.fm}, for example, allows a logged in user to set his theme by clicking the `paint it black' link at the top of the site.}. Because this theme can immediately be applied (by changing the page's css style on the fly), there is no need to send an extra HTTP request to refresh the page. To make this style change persistent, a cookie has to be set in the user's browser. To avoid the need for another HTTP request and response, this cookie is set using JavaScript. Subsequent requests will however send this cookie over HTTP, to make sure that the returned page is using the correct style. If no distinction is made between session cookies and other cookies, sending this cookie over HTTP would not be allowed by our policy. Correctly identifying which cookies are session cookies is the subject of the next section.

Similarly, the policy is only applied to session cookies set via an untrusted channel, that are sent over a trusted channel afterwards. Although we assume that session cookies which are requested over a trusted channel will never be set over an untrusted channel, session cookies can still be set over an untrusted channel and be requested over the same channel afterwards. We allow this type of behavior because we don't want to break web applications which set session cookies via JavaScript, only to subsequently request them via JavaScript. 

In normal use cases, setting session cookies via JavaScript will not be very common. Indeed, when a user logs in, he will most likely be redirected to either a personalized homepage, or to a personalized version of the web page he was reading before he logged in. Because the HTTP response needed to serve the personalized page is able to set new cookies, session cookies are almost always set over HTTP.

The proposed policy effectively mitigates the attack vectors we considered to be in scope. Cookies set from JavaScript are marked as untrusted, which mitigates the cross-site scripting attack vector, both within one domain as for sites sharing cookies across subdomains. A second attack vector is the injection of cookies through the \texttt{<meta>}-tag. Since these cookies do not come from a \texttt{Set-Cookie} header, they too are considered untrusted. As with the cross-site scripting attack vector, both attacks within one domain and across subdomains are mitigated. In section \ref{evaluation}, we show that dismissing untrusted session cookies has no impact on the user experience. The complete policy is summarized in table \ref{tab:nofix-policy}.

\begin{table}[htb]
	\centering
	\begin{tabular}{r|cc}
		& Regular Cookie & Session Cookie\\
		\hline
		Trusted Channel & Allowed & Allowed\\
		Untrusted Channel & Allowed & Not Allowed\\
	\end{tabular}
	\caption{The client-side policy for preventing session fixation}
	\label{tab:nofix-policy}
\end{table}

\section{Identifying session identifiers}\label{detecting-sids}

Because there is no standardized way of doing session management (see section \ref{session-management}), we have to look for patterns which are common among session identifiers. Based on the properties of secure web sessions listed in section \ref{secure-sessions}, and on the SID detection algorithm defined by Nikiforakis et al. \cite{Nikiforakis2010} (see section \ref{sessionshield}), we identify a cookie to be a session cookie if it has one of the following properties:
\begin{itemize}
	\item The name of the cookie is in a list of known session identifiers, such as \texttt{phpsessid}, \texttt{aspsessionid} and \texttt{jspsession}. Most frameworks implementing a session mechanism have default names for their session IDs, which are included in this list.
	\item The name of the cookie contains the substring `sess' and the value of the cookie is a sufficiently long\footnote{Based on \cite{Nikiforakis2010}, we consider 10 characters to be sufficiently long.} string containing both numbers and characters.
	\item The cookie value passes a randomness test. For this, the relative entropy of the cookie value is computed based on the Shannon Entropy measure \cite{Kaplan2002}. Afterwards, it is calculated how many bits would be needed to encode the string \cite{Nikiforakis2010}.
\end{itemize}

\section{Implementation}

The solution was first prototyped as a lightweight HTTP proxy written in Python, and later implemented as an add-on for Mozilla Firefox\footnote{This add-on is available for download at \url{https://spideroak.com/share/IJZGC3I/pub/home/bram/Openbaar/NoFix.xpi}. It can also be tested by booting the live CD included with this thesis, for which the instructions are available in Appendix \ref{manual}}. We first discuss the advantages and disadvantages of an implementation as a browser extension. Afterwards, we discuss the relevant parts of the Firefox architecture. We conclude with the specifics about the implementation itself.

\subsection{HTTP proxy or browser extension?}\label{proxy-or-extension}

Implementing the policy as a browser extension has numerous advantages over an implementation as an HTTP proxy. The most important advantage is visible when an encrypted (\gls{https}, see section \ref{ssl}) connection is used. In this case, a browser extension is already behind the regular SSL endpoint. An HTTP proxy, on the other hand, should provide its own SSL endpoint if it wants to intercept secured traffic. This means that it should handle encryption, handshakes and certificates -- which are all very difficult to get right -- separately from the browser. The other option is to not let the proxy intercept HTTPS traffic, rendering it incapable to protect this kind of traffic against session fixation. This would be a major security compromise, since SSL traffic is deemed to be more secure than normal (unencrypted) traffic.

A second advantage lies in the fact that an extension can differentiate between the normal and private browsing modes in the browser. Thus, it can make sure normal surfing cookies are never mixed with private surfing cookies. A proxy would allow cookies that were set during a private browsing session to pass through in a normal browsing session. Moreover, since a proxy has no way of distinguishing between the various applications using it, different browsers would be allowed to use each other's cookies.

An additional advantage is that a browser already has an up-to-date list of top level domain names. This is convenient when checking whether a website is trying to set a cookie for a valid parent domain (and not a top-level domain).

Lastly, installing a browser extension -- and especially a Firefox add-on -- is trivial, while installing a proxy requires changing the connection settings in all applications that should use the proxy. Thus, an implementation in the form of a browser extension allows us to reach a much broader -- and less technically inclined -- audience \cite{Bonne2011}.

There are also some disadvantages a browser extension has compared to a proxy. The most obvious one is that a specific extension implementation will only be usable by at most a few browsers. A proxy, on the other hand, can protect all HTTP traffic that passes through, protecting even applications that are not web browsers. Moreover, some browsers do not provide support for extensions at all, causing a proxy to be the only viable solution for these applications.

A second disadvantage is that, on some browsers, extensions are able to interfere with one another. In Firefox, for example, all extensions share the same namespace. This allows rogue browser extensions to thwart security extensions \cite{Barth2010,TerLouw2007}.

\subsection{The Firefox architecture}

Mozilla Firefox allows to extend the browser using a combination of JavaScript and XML. These can access the browser's XPCOM components, which offer access to various browser features.

In order to implement our client-side protection technique, the following capabilities are needed:
\begin{itemize}
	\item Inspect incoming HTTP(S) responses, to be able to track the trusted session identifiers that are being set.
	\item Inspect and modify outgoing HTTP(S) requests, to be able to strip out untrusted session identifiers.
	\item Persistently store information, to be able to keep information on persistent session identifiers.
\end{itemize}
These capabilities are provided by the following Firefox components:
\begin{itemize}
	\item The \texttt{http-on-examine-response} observer allows us to intercept HTTP responses before they are processed. Whenever a response is received, this object's \texttt{observe()} method is executed \cite{MozillaObservers}.
	\item The \texttt{http-on-modify-request} observer allows us to intercept and modify HTTP requests before they are sent. Whenever a request is received, this object's \texttt{observe()} method is executed \cite{MozillaObservers}.
	\item The \texttt{storageService} allows us to persistently store data in a SQLite database \cite{MozillaStorage}. This service is also used by Firefox internally to store data on the local machine \cite{Bonne2011,MozillaStorage}.
\end{itemize}

Additionally, Firefox provides an interface that examines a hostname and determines the longest portion that should be treated as though it were a top-level domain (\gls{tld}) \cite{MozillaDevelopers2010}. This is convenient when checking whether a cookie is being set for an allowed domain.

It must be noted that the previously listed `required capabilities' are currently only partially available in all other browsers. Google Chrome, for example, had no support for intercepting HTTP requests and responses at the time of writing, although this feature is currently on the wishlist \cite{ChromiumWishlist}. To be able to port the add-on to other browsers, these browsers should implement similar interfaces.

\subsection{Implementation as a Firefox add-on}

The proposed policy was implemented as a Firefox add-on, available for Firefox 3.5 or higher. We describe the internals of the add-on in this section.

To detect when a trusted session ID is set, the add-on searches for a \texttt{Set-Cookie} header in all incoming HTTP responses. If a cookie contains the \texttt{domain} attribute, the value of this attribute is checked for validity in order to prevent cross-site cooking attacks \cite{Zalewski2006} against the add-on. For this, the add-on assures that \texttt{domain} parameters in cookies are always parent domains or subdomains of the website the cookie was received from. Also, Firefox' top-level domain list \cite{MozillaDevelopers2010} is used to make sure no website can set a cookie for a top-level domain (such as \emph{.co.uk}).

When all requirements are satisfied, the cookie is stored in a separate cookie jar implemented as a SQLite database. Storage is handled using asynchronous writes, to reduce the delays introduced by the add-on (we will discuss the add-on's performance in section \ref{performance}). To make sure that session IDs are also available for checking requests before their write operation is complete, cookies are temporarily stored in main memory until they have been written to the database.

To filter out untrusted session IDs from requests, every outgoing HTTP request is intercepted by the add-on. Before a request is sent to the server, its \texttt{Cookie} header is inspected. For each cookie, the add-on checks whether the cookie represents a session identifier using the algorithm described in section \ref{detecting-sids}. If it does, the add-on's separate cookie jar is queried to make sure that the cookie is trusted. If this is not the case, this particular cookie is stripped from the request. After having repeated this process for all cookies, the request is released back to the browser with the modified \texttt{Cookie} header in place.

\section{Evaluation}\label{evaluation}

The add-on is evaluated in three parts. The first part evaluates the add-on's functional correctness. The second part evaluates its impact in real-world browsing scenarios. Lastly, the third part evaluates the add-on's performance, and thus -- together with the second part -- its impact on the user experience.

\subsection{Scenario testing}

To make sure that the implementation behaves in accordance with the specification, a simple HTTP server is created. This server simulates different scenarios by issuing cookies both in the way a session fixation attack would, and in the way a legitimate website would. It is then checked that only the trusted session identifiers are returned to the server on subsequent requests, while untrusted session identifiers are stripped. Our implementation passes all these correctness tests.

\subsection{Real-world impact}

An evaluation of the add-on's impact on the user experience is difficult to automate. Because of this, we asked several test persons to use the add-on during their daily web browsing activities. Apart from collecting logging data from the test persons, we asked them to pay close attention to see if any websites would break or behave differently. The experiment ran for 20 weeks.

From the log files, we find that in a considerable fraction (19.53\%) of the requests, session cookies were stripped. We would assume that, with such statistics, the extension leads to a severely degraded user experience. Surprisingly, the only problems that were mentioned by the testers had other causes, such as malfunctioning websites or other poorly written add-ons. Not a single problem mentioned in these 20 weeks was due to the policy applied by our add-on. We will discuss the causes for this behavior in section \ref{discussion}.

Because of the very little real-world impact of this solution, it would even make sense for Firefox to implement the described policy by default. In the past, the Firefox team has already taken bold steps to increase security, even though some websites might have been broken afterwards \cite{Singh2010,MozillaXmlHttp,Bonne2011}.

\subsection{Performance}\label{performance}

The user experience can also be negatively impacted by a large overhead induced by client-side protection mechanisms \cite{Bonne2011}. We quantify this performance overhead by performing two experiments.

The first experiment consists of timing the delays that are introduced by the add-on during an everyday web browsing session. For this, timers are added to the add-on code. We measure that the average delays for requests (where it is checked whether a cookie is allowed) and responses (where cookies are set) are 1.25\ ms and 0.25\ ms, respectively.

The second experiment\footnote{This experiment was performed by Nick Nikiforakis, co-author of the submission to W2SP 2011 \cite{Bonne2011}.} that is performed compares the \emph{page} load times in Firefox with and without the add-on enabled for the top 1000 most visited pages on the Internet, according to Alexa \cite{Alexa1000}. For this experiment, we use a setup similar to the one used to evaluate SessionShield \cite{Nikiforakis2010}, where the network inconsistencies are eliminated by hosting all pages to be requested locally. A fake local DNS server and fake local web server are used to serve the captured pages. For additional resources (e.g. images, scripts), the local web server responds with a 404 `page not found' status code. The actual experiment consists of using the ChickenFoot add-on \cite{ChickenFoot} to automate the process of starting a timer, instructing Firefox to load a page, and stopping the timer. This scenario is repeated three times for every web page, both with and without the add-on enabled. The average page-load time without session fixation protection is 195.407\ ms. When our add-on is enabled, the average page-load time is 198.242\ ms \cite{Bonne2011}. Thus, the average overhead introduced by our add-on is approximately 3\ milliseconds, which is negligible compared to the more than 0.5\ seconds average loading time for pages on the Internet \cite{Webmetrics}.

\section{Discussion}\label{discussion}

Our client-side countermeasure against session fixation is very effective with no impact on the user experience and a minimal impact on the site's behavior in the form of false positives (i.e. regular cookies which are incorrectly marked as session cookies) \cite{Bonne2011}. In this section, we describe why the add-on blocks so many cookies, and why this does not affect the web application's behavior or the user experience.

One reason is that a web application might not need cookies that are set via JavaScript to be available in HTTP requests. Indeed, when a cookie is set via JavaScript, chances are quite high that this cookie will afterwards only be accessed via JavaScript. However, a JavaScript cookie will also always be sent over HTTP by the browser, causing it to be stripped by the add-on. This does not impede the web application's behavior because the web server would not read the cookie via HTTP anyway. When the web server tries to access the cookie via JavaScript, however, it is allowed to do so by our policy. Thus, a web application only accessing untrusted cookies via JavaScript will keep working as expected.

Another reason for the stripped cookies is that the add-on also exhibits some false positives \cite{Bonne2011}. False positives are cookies that should be allowed to be sent over HTTP, but which were erroneously stripped from the request by the add-on. The cookies that we discovered to be treated as session cookies, even though they were not, are all related to web analytics services. Web analytics services provide a way for web developers to gather statistics about their website. Two examples of such services are Google Analytics\footnote{An overview of the features available in Google Analytics can be found on \url{http://www.google.com/analytics/features.html}.} and Yahoo! Web Analytics\footnote{An overview of the features available in Yahoo! Web Analytics can be found on \url{http://web.analytics.yahoo.com/features}.}. To be able to track a visitor's behavior on a website, these services set cookies to measure the time spent on the website, and to uniquely identify a visitor \cite{Tappenden2009,GoogleAnalytics}. This last category of cookies often has a value that resembles a session ID. Because the web analytics code is embedded as JavaScript code within the developer's web page, a web analytics cookie is set via JavaScript. Hence, these user identification cookies will be marked as `untrusted' by the add-on, and will as such be stripped from all HTTP requests. Stripping these cookies only affects the web developer, who loses some statistical information about visitors using our add-on. It does not affect the user experience, which is why none of the add-on testers noticed any problems.

If we want our policy to be widely adopted, we should make sure that the previously mentioned false positives are accounted for, even if they don't impede the user experience. Because of this, we also provide a whitelist of cookies that should not be stripped in the add-on.

An approach using heuristics will also always exhibit some false negatives. However, when an SID is incorrectly marked as a regular cookie in our session ID detection mechanism, this means that it does not satisfy the safety requirements for SIDs presented in section \ref{unguessable} \cite{Nikiforakis2010}. Because of this, the SID would already be susceptible to being guessed by an attacker, which will still be a problem even if the user would be secure against session fixation attacks on a website that uses such SIDs. When the web developer fixes the problem of insecure session IDs, the session fixation problem will automatically be solved by our solution.

\section{Extending the solution with session hijacking protection}

As we already mentioned in the introduction, the add-on was extended with a solution to the session hijacking attack. To do this, we made an even stronger distinction between HTTP cookies and JavaScript cookies: in our extended policy, session cookies that are set via HTTP are not allowed to be accessed by JavaScript. This prevents session cookies from being stolen via \gls{xss} attacks. This policy closely resembles that of SessionShield \cite{Nikiforakis2010}, the client-side solution to session hijacking attacks which was described earlier. A comparison of both solutions is given in section \ref{sessionshield-comparison}.

The policy extension is implemented in our add-on by adding the \texttt{HttpOnly} flag to every incoming HTTP cookie which is marked as a session cookie. When this flag is enabled, the browser will only allow the cookie to be read via HTTP (see section \ref{httponly} for more information). Thus, we let the browser handle the blocking of HTTP session cookies over JavaScript. Figure \ref{fig:clientside-httponly} shows the adapted version of Figure \ref{fig:clientside-request}, with the solution extended to provide session hijacking protection.

\begin{figure}[htb]
	\centering
	\includegraphics[width=.7\textwidth]{img/clientside-proxy-3.png}
	\caption[The extended client-side solution]{The extended client-side solution, providing both session fixation and session hijacking protection}
	\label{fig:clientside-httponly}
\end{figure}
 
\section{Comparison to other session attack countermeasures}\label{related-work}

In this section, we compare our solution to some of the countermeasures discussed in the previous chapter.

\subsection{Renewing the session identifier}

Renewing the session identifier was mooted to be the ideal solution against session fixation attacks in the previous section. Unfortunately, this solution can only be implemented at the server side. Because of this, Johns et al. suggest that their proxy protecting against session fixation could also be implemented at the client side, renewing the PSID instead of the SID to protect the user \cite{Johns2011}. It would then be expected that the proxy makes sure that an attacker can not take over a victim's session by executing a session fixation attack, since he does not know the victim's PSID. Unfortunately, this is not the case. To see why, consider the scenario depicted in Figure \ref{fig:johns-clientside}. In this scenario, the attacker uses HTTP response splitting (as described in section \ref{splitting}) to inject a cookie. Because the proxy sees the SID as a normal HTTP session cookie, he will attach a valid PSID to the cookie before sending it along to the victim. When the victim logs in, he attaches both the SID and the PSID, of which the proxy will only forward the first one to the web server. The victim is now logged in with the fixated SID, as before. Since the attacker's requests do not have to pass through the proxy, he can use the fixated SID to act as the victim on the server.

\begin{figure}[htb]
	\centering
	\includegraphics[scale=0.4]{img/johns-attack.png}
	\caption[Attack on a client-side implementation of Johns et al.]{Attack on a client-side implementation of the proxy proposed by Johns et al.}
	\label{fig:johns-clientside}
\end{figure}

Note that in the previous discussion, we only talked about injecting a SID via HTTP response splitting. The reason for this is that the proxy has been developed specifically for HTTP cookies, which will prevent it from detecting cookies being set via other channels such as JavaScript or \texttt{<meta>} tags. In fact, this causes a client-side implementation of the proxy to effectively act like our own solution: session cookies which have been set over a channel different from HTTP will not have a valid PSID attached to them, and will as such be stripped from the requests by the proxy.

It is not possible for a client-side proxy to be configured for every possible web application. Because of this, the paper argued that such a solution should be able to automatically identify session cookies. Our solution uses a SID detection algorithm exactly for this purpose.

\subsection{HttpOnly}\label{httponlyremark}

Apart from being used as the mechanism to make our solution able to prohibit JavaScript access to HTTP session cookies, HttpOnly greatly influenced the development of our session fixation solution. Indeed, our solution can be seen as the dual of HttpOnly cookies: where the \texttt{HttpOnly} flag makes sure that HTTP cookies are unavailable to untrusted channels, our solution makes these untrusted channels unable to set cookies which will be sent over HTTP.

A difference between HttpOnly and our solution is that the \texttt{HttpOnly} flag must be explicitly set at the server side, whereas our solution reasons autonomously about which cookies should be made unavailable.  Moreover, our policy extension could be seen as a proxy that automatically enforces HttpOnly behavior for cookies that need it.

\subsection{Deferred loading and one-time cookies}

In the deferred loading approach, proposed by Johns et al. \cite{Johns2006}, it is reasoned that cookies should be kept separate from the page content, to prevent an attacker from stealing a cookie by altering the page. Similarly, one-time cookies \cite{Dacosta2011} are never made available to JavaScript. Our approach is similar in that it keeps session cookies unavailable to JavaScript. A difference is that our approach identifies session cookies after they have been issued by the web server, and thus does not require cooperation of the server.

\subsection{SessionShield}\label{sessionshield-comparison}

The policy extension, wherein the default policy was augmented with session hijacking protection, is closely related to the session hijacking solution proposed by Nikiforakis et al. \cite{Nikiforakis2010}. In this solution, HTTP session cookies are kept in a separate cookie store, outside of the browser. The difference lies in the fact that our add-on \emph{does} allow the browser access to the cookies: it only prevents non-HTTP channels from accessing session cookies. As was discussed in section \ref{proxy-or-extension}, an implementation in the form of a browser extension makes sure that different browsers will not be able to use each other's cookies. This solves the problem that the SessionShield proxy has when it is used by different browsers.

Similarly to our solution, SessionShield does allow untrusted cookies to be accessed via untrusted channels afterwards. This is because, like our solution, SessionShield only intercepts HTTP traffic, and does not notice JavaScript and \texttt{<meta>} cookies being set. This allows JavaScript cookies to keep functioning correctly.

As we already noted in the previous chapter, SessionShield does not protect against session fixation. However, it would be relatively easy to extend SessionShield with our policy preventing session fixation. This can be achieved by making SessionShield block session cookies that it receives from the browser. Indeed, since SessionShield only intercepts HTTP traffic, it keeps in its cookie store only cookies that are set via HTTP. Thus, when a cookie that SessionShield does not know of is received from the browser, it can conclude that this cookie was set over an untrusted channel. All SessionShield has to do then is use its SID detection algorithm (which it shares with our solution) to check whether the cookie is a session cookie. If it is, SessionShield can block the cookie, thus mitigating session fixation attacks.

\subsection{Noxes and dynamic tainting}

The goal of both the Noxes and dynamic tainting solutions is very similar to that of our own solution: to prevent session identifiers from leaking via \gls{xss} attacks \cite{Kirda2006,Vogt2007}. Their approach, however, is rather different. Both Noxes and dynamic tainting try to prevent the stealing of cookies by mitigating all attack vectors one by one. Our extended solution takes this one step further by making session cookies completely unavailable to channels other than HTTP. This makes sure that hidden vectors, which were a problem with both Noxes and dynamic tainting, are accounted for.

\section{Future work}

While cross-domain cooking \cite{Zalewski2006} can also be used to perform a successful session fixation attack, it was largely ignored in our solution. Since our Firefox add-on uses Firefox' top-level domain list to determine whether a cookie is set for a top-level domain, part of this problem is already solved. However, a complete client-side solution to cross-domain cooking would have to consist of completely prohibiting subdomains from setting cookies for their parent domain. Our logs show that, when loading each of the homepages of Alexa's top 100 websites only once, 427 cookies are set for a parent domain. Because of this, it is currently not clear how such a solution can be implemented without breaking lots of websites.

Other session fixation attacks make use of URL parameters and form fields to set the session cookie. These were also not considered in the current implementation. A simple countermeasure could consist of stripping all SIDs from URLs and POST data. Unfortunately, this would break all websites that perform their session management exclusively via URL rewriting. Nevertheless, it might be useful to test the real-world impact of such a solution, since statistics gathered from the K.U.Leuven dataset (containing 4.7 million requests to 11848 domains) indicate that only 0.3\% of the URLs contain session information created by a known web framework \cite{DeRyck2010,Bonne2011}.

The current solution was only prototyped as an HTTP proxy and implemented as a Firefox extension. Similar extensions could be written for other browsers. Moreover, as was already described in section \ref{sessionshield-comparison}, SessionShield could easily be extended with our solution. This would yield a proxy with a policy similar to our own, that separates session cookies entirely from the browser.

\section{Conclusion}

In this chapter, we developed a client-side countermeasure to both session fixation and session hijacking attacks. We reasoned that users are likely to be more willing to protect themselves against session attacks than web developers are to protect their web application, hence the need for a solution at the client side.

The principle of the solution is that a clear separation should be made between HTTP cookies and cookies set via other channels. In this sense, our approach is very similar to that of SessionShield and HttpOnly cookies. However, our approach extends this behavior to make it robust against session fixation attacks.

The solution was implemented as an add-on for the Firefox web browser, which allowed it to be used by testers for a period of 20 weeks. We tested the impact of enforcing a policy that prevents untrusted cookies from being sent over HTTP (the part of the add-on which protects against session fixation), and found that -- even though a considerable amount of cookies is blocked -- enforcing such a policy has no negative effect on the user experience. Moreover, the implementation of this policy has no noticeable impact on web browsing performance.
