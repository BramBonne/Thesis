\chapter{Introduction} %TODO

Over the last few years, the web has shifted from being a collection of pages containing static information to a dynamic and fully interactive platform. Where the internet was once used only as an information repository, today it powers complex web applications, developed both to replace programs that were once running locally on a user's computer, and to provide whole new functionality that is possible only on the web. For this, web protocols to be used and extended in ways they were never imagined to be.

More than in the past, these web applications deal with sensitive personal information. Thanks to the emergence of web-based mail applications akin to Google's GMail and Microsoft's Hotmail, and social networks like Facebook and Netlog, a great deal of user information is stored on the servers of web applications. Moreover, shops have moved to the online world, and payments can be made online by using online banking or credit cards. Because many web applications handle such sensitive information, security of a user's information and identity is of utmost importance.

User authentication is handled in most web applications via the concept of web sessions. These allow users to use a web application without having to enter their login credentials for every action. Unfortunately, sessions have many security weaknesses. OWASP, a leading organization in the field of web application security, rates `Broken Authentication and Session Management' as the third most important web application security risk \cite{Williams2010}.

In this thesis, the security of sessions in web applications is examined.

% Johns2011 provides the first work about session fixation solution, and acknowledges its importance. We created the first client-side countermeasure against session fixation.
