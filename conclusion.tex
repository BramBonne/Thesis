\chapter{Conclusion}

In this thesis, we looked at the security of session management in web applications.

We started off by showing the different methods of session handling that exist on the web today. We discussed how session identifiers can be accessed, and which properties a secure session identifier should possess.

Since session management is essentially an ad hoc approach, not developed with security in mind, it is prone to exploitation by an attacker. Indeed, we saw that `Broken Authentication and Session Management' is very high on OWASP's list of web application vulnerabilities. In the third chapter, we described attacks that make an attacker able to act on a web server as if he was another user. We showed that these attacks are very applicable in the present dynamic web, which makes great use of user management and user generated content. While the attacks themselves are not very complicated, many attack vectors are available. This allows an attacker to create complex attack scenarios which are difficult to prevent.

A literature study showed that many countermeasures to session attacks have been proposed over the years. Solutions exist both at the client and at the server side: some offer protection against specific attacks, while others try to improve session security in general. Thoroughly examining these solutions showed that, while most of them suffer shortcomings, some good countermeasures to session attacks are available: especially popular web frameworks provide reasonably good protection. A solution offering client-side protection against session fixation attacks is, however, inexistent. As a consequence, users have to rely on the developers of all web applications they are using to implement this protection at the server side. To solve this problem, we developed our own client-side solution to session fixation attacks.

Our proposed client-side solution provides protection against both session fixation and session hijacking attacks, where an attacker uses untrusted channels to access or modify the session cookie. We implemented this solution as an add-on for the Firefox web browser. An extensive evaluation of the add-on showed that, while our solution has no negative impact on the user experience, it does prevent session fixation and session hijacking via one of these untrusted channels. This allows a user to protect himself against session hijacking and session fixation attacks without requiring the web developer to secure his web application.
