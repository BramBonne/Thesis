\chapter{Conclusion}

In this thesis, we looked at security in web applications that make use of sessions to manage user authentication.

We started by showing the different methods of session handling that exist on the web today. We discussed how session identifiers can be accessed, and which properties a secure session identifier should possess.

Since session management is essentially an ad hoc approach, not developed with security in mind, it is prone to exploitation by an attacker. Indeed, we saw that `Broken Authentication and Session Management' is very high on OWASP's list of web application vulnerabilities. In the third chapter, we described attacks that make an attacker able to act on a web server as if he was another user. We showed that these attacks are very applicable in the present dynamic web, which makes great use of user management and user generated content. While the attacks themselves are not very complicated, many attack vectors are available. This allows an attacker to create complex attack scenarios which are difficult to prevent.

We saw that, over the years, many countermeasures to session attacks have been proposed. Solutions exist both at the client and server side, for specific attacks as well as for session security in general. While many of these solutions suffer shortcomings, some good countermeasures to session attacks are available: especially popular web frameworks provide reasonably good protection against session attacks. One exception is that of client-side protection against session fixation attacks, which is why we developed our own client-side solution to these attacks.

Our proposed client-side solution provides protection against both session fixation and session hijacking attacks, where an attacker uses untrusted channels to access or modify the session cookie. We showed that, while our solution has no negative impact on the user experience, it does prevent session fixation and session hijacking via one of these untrusted channels.
